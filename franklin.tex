\documentclass{patmorin}
\listfiles
\usepackage[utf8]{inputenc}
\usepackage{microtype}
\usepackage{amsthm,amsmath,graphicx}
\usepackage{pat}
\usepackage[letterpaper]{hyperref}
\usepackage[table,dvipsnames]{xcolor}
\definecolor{linkblue}{named}{Blue}
\hypersetup{colorlinks=true, linkcolor=linkblue,  anchorcolor=linkblue,
citecolor=linkblue, filecolor=linkblue, menucolor=linkblue,
urlcolor=linkblue} 
\setlength{\parskip}{1ex}
\usepackage{wasysym}

\title{\MakeUppercase{Notes and Questions on the Expected Performance 
   of the Franklin Leader Election Algorithm}%
   \thanks{This work was partly funded by NSERC and the Ontario Ministry of
    Research, Innovation and Science}}

\author{Probability and Combinatorics 2018}

%\usepackage[mathlines]{lineno}
%\linenumbers
%\setlength{\linenumbersep}{2.5cm}
%\rightlinenumbers
%\linenumbers
%\newcommand*\patchAmsMathEnvironmentForLineno[1]{%
%  \expandafter\let\csname old#1\expandafter\endcsname\csname #1\endcsname
%  \expandafter\let\csname oldend#1\expandafter\endcsname\csname end#1\endcsname
%  \renewenvironment{#1}%
%     {\linenomath\csname old#1\endcsname}%
%     {\csname oldend#1\endcsname\endlinenomath}}% 
%\newcommand*\patchBothAmsMathEnvironmentsForLineno[1]{%
%  \patchAmsMathEnvironmentForLineno{#1}%
%  \patchAmsMathEnvironmentForLineno{#1*}}%
%\AtBeginDocument{%
%\patchBothAmsMathEnvironmentsForLineno{equation}%
%\patchBothAmsMathEnvironmentsForLineno{align}%
%\patchBothAmsMathEnvironmentsForLineno{flalign}%
%\patchBothAmsMathEnvironmentsForLineno{alignat}%
%\patchBothAmsMathEnvironmentsForLineno{gather}%
%\patchBothAmsMathEnvironmentsForLineno{multline}%
%}


\newcommand{\question}[1]{\textbf{\color{red}Question:}~#1}

\DeclareMathOperator{\pw}{pw}

\newcommand{\eps}{\epsilon}



%\pagenumbering{roman}
\begin{document}
%\begin{titlepage}
\maketitle
%
\begin{abstract}
  We present upper and lower-bounds on an algorithm of Franklin for
  leader election on the ring.
\end{abstract}
%\end{titlepage}
%
%\tableofcontents
%
%\newpage


\section{Introduction}
\pagenumbering{arabic}

Franklin's leader election algorithm is a classic algorithm for finding
the maximum value on an $n$-node ring network (a cycle) whose nodes are
have distinct labels $x_0,\ldots,x_{n-1}$ in the order they appear around
the cycle.  The algorithm is initialized with all nodes in an active
state and proceeds in rounds.  During each round, each node $i$ gets the
labels $x_{i'}$ and $x_{i''}$ of its nearest active neighbour in
the counterclockwise and clockwise directions and then checks if $x_i <
\max\{x_{i'},x_{i''}\}$ and, if so, node $i$ becomes inactive and does
not partcipate in any subsequent rounds.  The algorithm terminates when
only one node remains active.

This algorithm clearly terminates after at most $\log_2 n$ rounds because
at most one out of every two consecutive active nodes remains active
after each round.  If the labels $x_0,\ldots,x_{n-1}$ are independent
uniformly distributed random variables over the interval $[0,1]$, then
it seems plausible that the expected number of rounds in this case is
$\log_3 n$.  This seems plausible because, during the first round, node
$i$ remains active if and only if $x_i=\max\{x_{i-1},x_i,x_{i-2}\}$ and
this event clearly has probability $1/3$.  Unfortunately, this intuition
can not easily be made rigorous and is, in fact, incorrect. After the
first round, the distribution of labels of surviving nodes is no longer
independent. Furthermore, a careful calculation (see \secref{X}) shows
that the probability that a particular node $i$ remains active after 2
rounds approaches
\[
    p_2 = \frac{3e^4 - 48e^2 + 233}{384} \approx 0.1096868681 < 1/9 \enspace .
\]


\bibliographystyle{plainurl}
\bibliography{anagram}

\end{document}


