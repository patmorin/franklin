\documentclass{patmorin}
\listfiles
\usepackage[utf8]{inputenc}
\usepackage{microtype}
\usepackage{amsthm,amsmath,amsopn,graphicx}
\usepackage{pat}
\usepackage[letterpaper]{hyperref}
\usepackage[table,dvipsnames]{xcolor}
\definecolor{linkblue}{named}{Blue}
\hypersetup{colorlinks=true, linkcolor=linkblue,  anchorcolor=linkblue,
citecolor=linkblue, filecolor=linkblue, menucolor=linkblue,
urlcolor=linkblue} 
\setlength{\parskip}{1ex}
\usepackage{wasysym}

\DeclareMathOperator*{\argmax}{arg\,max}
\DeclareMathOperator{\cw}{succ}
\DeclareMathOperator{\ccw}{pred}

\title{\MakeUppercase{Notes and Questions on the Expected Performance 
   of Franklin's Leader Election Algorithm}%
   \thanks{This work was partly funded by NSERC and the Ontario Ministry of
    Research, Innovation and Science}}

\author{Probability and Combinatorics 2018}

%\usepackage[mathlines]{lineno}
%\linenumbers
%\setlength{\linenumbersep}{2.5cm}
%\rightlinenumbers
%\linenumbers
%\newcommand*\patchAmsMathEnvironmentForLineno[1]{%
%  \expandafter\let\csname old#1\expandafter\endcsname\csname #1\endcsname
%  \expandafter\let\csname oldend#1\expandafter\endcsname\csname end#1\endcsname
%  \renewenvironment{#1}%
%     {\linenomath\csname old#1\endcsname}%
%     {\csname oldend#1\endcsname\endlinenomath}}% 
%\newcommand*\patchBothAmsMathEnvironmentsForLineno[1]{%
%  \patchAmsMathEnvironmentForLineno{#1}%
%  \patchAmsMathEnvironmentForLineno{#1*}}%
%\AtBeginDocument{%
%\patchBothAmsMathEnvironmentsForLineno{equation}%
%\patchBothAmsMathEnvironmentsForLineno{align}%
%\patchBothAmsMathEnvironmentsForLineno{flalign}%
%\patchBothAmsMathEnvironmentsForLineno{alignat}%
%\patchBothAmsMathEnvironmentsForLineno{gather}%
%\patchBothAmsMathEnvironmentsForLineno{multline}%
%}


\newcommand{\question}[1]{\textbf{\color{red}Question:}~#1}

\DeclareMathOperator{\pw}{pw}

\newcommand{\eps}{\epsilon}



%\pagenumbering{roman}
\begin{document}
%\begin{titlepage}
\maketitle
%
\begin{abstract}
  We present upper and lower-bounds on the performance of an algorithm
  of Franklin for leader election on the ring when the node identifiers
  are independent uniform random variables in $[0,1]$.
\end{abstract}
%\end{titlepage}
%
%\tableofcontents
%
%\newpage


\section{Introduction}
\pagenumbering{arabic}

Franklin's leader election algorithm is a classic algorithm for finding
the maximum value on an $n$-node ring network (a cycle) whose nodes
are have distinct labels $x_0,\ldots,x_{n-1}$ in the order they appear
around the cycle.  The algorithm is initialized with all nodes in an
active state and proceeds in rounds.  During each round, each node
$i$ sends and its label $x_i$ to its nearest active neighbour $i'$ in
the clockwise direction and its nearest active neighbour $i''$ in the
counterclockwise direction.  At the same time node $i$ receives the values
of the labels $x_{i'}$ and $x_{i''}$.  Node $i$ then checks if $x_i <
\max\{x_{i'},x_{i''}\}$ and, if so, node $i$ becomes inactive and does
not partcipate in any subsequent rounds, except to forward messages on
behalf of active nodes.  The algorithm terminates when only one node
remains active.

This algorithm clearly terminates after at most $\log_2 n$ rounds because
at most one out of every two consecutive active nodes remains active after
each round.  If the labels $x_0,\ldots,x_{n-1}$ are independent uniformly
distributed random variables over the interval $[0,1]$, then it seems
plausible that the expected number of rounds in this case is $\log_3 n$.
This seems plausible because, during the first round, node $i$ remains
active if and only if $x_i=\max\{x_{i-1},x_i,x_{i-2}\}$ and this event
clearly has probability $1/3$. Thus, in this probabilistic setting,
the expected number of nodes that remain active after the first round
is $n/3$.

Unfortunately, this argument fails in subsequent rounds because,
after the first round, the labels of active nodes are no longer
independent. Furthermore, a careful calculation (see \secref{p2}) shows
that the probability that a particular node $i$ remains active after 2
rounds approaches
\[
    p_2 = \frac{3e^4 - 48e^2 + 233}{384} \approx 0.1096868681 < 0.111\ldots = 1/9 \enspace .
\]
This is somewhat surprising and suggests that the Franklin leader election algorithm may perform better than expected when identifiers are random.  In particular, it is conceivable that the expected number of rounds in this setting could be $o(\log n)$.


\subsection{The Model and Results}

For a set $S\subseteq\Z$ and any $i\in\Z$, define $\ccw_S(i) =
i-\min\{j\in\N: i-j\in S\}$ and $\cw_S(i) = i+\min\{j\in\N: i+j\in S\}$.
We define $\cw_S^0(i)=\ccw_S^0(i)=i$ and, for $k\in\N$ $\cw_S^k(i) =
\cw_S(\cw_S^{k-1}(i))$ and $\cw_S^{-k}(i) = \ccw_S^{-(k-1)}(i)$.
For a set $S\subseteq\Z$, $\langle X_i: i\in S\rangle$ denotes the
sequence in which, for each $i\in S$, the value $X_i$ immediately precedes
the value $X_{\cw_S(i)}$.

To understand Franklin's algorithm as $n\to\infty$, we consider the infinite
case in which the nodes consist of all integers $i\in \Z$ and the labels are
independent uniform random variables $X_i\in[0,1]$, for all $i\in\Z$.

The initial set $A_0$ of active nodes is defined to be $A_0=\Z$ and, at round $r\ge 1$, we determine the set $A_r$ of active nodes after round $r$ as 
\[
    A_r = \{i \in A_{r-1} : i = \max\{X_{\ccw_{A_{r-1}}(i)},X_i,
                                      X_{\cw_{A_{r-1}}(i)}\} \} 
\] 
We study the probability $p_r$ that, for some particular $i\in\Z$,
$i\in A_r$.  Intuitively, one would expect that $p_r=1/3^r$, but the
we know that this is not even true for $r=2$.  Relaxing this, we might
hope to prove that $\lim_{r\to\infty} p_r^{1/r} = c$ for some $c$ close
to $1/3$.  However, \emph{apriori}, it is not obvious that this limit
exists and, if it does, it may be as large as $1/2$ or as small as $0$.

Here we establish the following upper and lower bounds on $p_r$:

\begin{thm}
   $A^r \le p_r \le B^r$.
\end{thm}

We have not been able to establish the existence of the limit
$\lim_{r\to\infty} p_r^{1/r}$ but this result shows that, if
$\lim_{r\to\infty} p_r^{1/r}=c$, then $0 < A \le c\le  B < 1/2$.

In terms of the number of rounds required by Franklin's algorithm on
a finite ring, these results establish that the expected number of
rounds is at most $\log_{1/B} n + o(\log n)$ and at least $\log_{1/A}
n- o(\log n)$.  In particular, these results rule out the possibility
that the expected number of rounds is $o(\log n)$.

\section{Derivation of $p_2$}
\seclabel{p2}

Svante's Derivation of $p_2$ can go here.

\section{Tools for Sequences}

The technical difficulty in studying Franklin's algorithm is that the
random variables $\langle X_i:i\in A_r\rangle$ are not independent.  In order to
address this difficulty, we study different sets, $L_r\subseteq
A_r\subseteq U_r$ that have the property that the sets $\langle X_i:i\in L_r\rangle$
and $\langle X_i:i\in U_r\rangle$ are each independent sets of random variables, which
makes them easier to study. In this way we obtain upper and lower bounds because
\[
    \Pr\{i\in L_r\} \le \Pr\{i\in A_r\} \le \Pr\{i\in U_r\} \enspace .
\]
To make this work, we need some preliminary results on the structure of
the sets we are working with.

In the following, all subsets of $\Z$ are infinite and have no maximum
or minimum.  Let $X=\langle X_i:i\in \Z\rangle$ be a set of distinct numbers indexed
by integers.  We say that a subset $S\subseteq\Z$ is \emph{$X$-safe}
if, for every $i\in S$ and
\[
    \max\{X_i,\cw_S(i)\} = \max\{X_m: m\in \{i,\ldots,\cw_S(i)\}\} \enspace .
\]

\begin{lem}\lemlabel{domination}
   Let $L\subseteq U\subseteq\Z$ be sets of integers where $L$
   is $X$-safe.  If $i\in L\cap U$ and $X_{\cw_U(i)} > X_i$, then
   \[
         X_{\cw_L(i)}\ \ge X_{\cw_U(i)} \enspace . 
   \]
   (And the same result holds if we replace all occurrences of $\cw$
   with $\ccw$.)
\end{lem}

\begin{proof}
   Let $j = \cw_U(i)$ and $k=\cw_L(i)$, so $i < j \le k$.
   Since $L\subset U$, $j\in \{i,\ldots,k\}$.  Since $L$ is $X$-safe,
   $\max\{X_i,X_k\} \ge \max\{X_i,\ldots,X_k\} \ge X_j$.  Now we have
   that $\max\{X_i,X_k\} \ge X_j$, but $X_i < X_j$, so it must be that
   $X_k\ge X_j$.
\end{proof}

For any set $A\subset\Z$ and $X=\{X_i:i\in \Z\}$, define the
\emph{Franklin function}
\[
   F_X(A) = \{ i\in A : X_{i} = \max\{X_{\cw_A(i)},X_i,X_{\ccw_A(i)}\} \}
\]
Notice that for any $r\ge 0$, the set $A_r=F^r_X(\Z)$ is $X$-safe.
With these definitions, the probabilities are are interested in are,
for some fixed $i$, $\Pr\{i\in A_r\}$ for some $r\in\N$.

\section{Lower Bounds}

We begin by proving the lower bound $p_r \ge 1/4^r$ by describing a
thresholded version of the Franklin function, $F_t$ that produces a
subset of the set $F$ produced by the Franklin function. $F_t$ is a
combination of two kinds of operations: thresholding and max-partitioning.

\paragraph{Thresholding:}
Let $L\subseteq\Z$ be an $X$-safe set.  Observe that, for any real
value $t$, the \emph{threshold set} $L_{\ge t}=\{i:X_i\ge t\}$ is also an
$X$-safe set. Indeed, for any $i\in L_{\ge t}$, $\max\{X_i,X_{\cw_{L_{\ge t}}(i)}\} = \max\{i,\ldots,\cw{L_{\ge t}}(i)\}$ since $X_i\ge t$, $X_{\cw_{L_{\ge t}}(i)}\ge t$ and $X_{i+1},\ldots,X_{\cw_{L_{\ge t}}(i)-1}$ are all less than $t$.

\paragraph{Max-Partitioning:}
Let $P$ be any partition of $L$ with the property that, for each $I\in
P$, if $i,j,k\in L$ with $i < k$ and $i,k\in I$, then $j\in I$. Then it
is straightforward to verify that the set
\[
     L_{P} = \bigcup_{I\in P} \argmax_{i\in I}(X_i)
\]
is also an $X$-safe set. Indeed, for any $i\in L_{P}$, there exists a $j\in\{i,\ldots,\cw{L_P}(i)-1\}$ such that $X_i=\max\{X_i,\ldots,X_j\}$ and $X_{\cw{L_P}(i)}=\max\{X_{j+1},\ldots,X_{\cw_{L_P}(i)}\}$

\paragraph{Thresholded Franklin:}
Consider the graph $G=(L,E)$ where $E=\{(i,\cw_L(i)): i\in L\}$. Remove
from $G$ all vertices $i\in L$ such that $X_i < t$ to obtain a graph
$G'$ that has a set $P$ of connected components.  Now consider the
\emph{$t$-thresholded Franklin function}
\[
    F_t(L) = \bigcup_{I\in P} \argmax_{i\in V(I)}(X_i)
\]
Note that, for any $t\in\R$, $F_t(L)$ is $X$-safe, since it can be
written as $F_t(L) = (L_{\ge t})_P$.

For any non-negative integer $r$ and any real values $t_1,\ldots,t_r$,
define $F_{t_1,\ldots,t_r}(L) = F_{t_r}(F_{t_{r-1}}(\cdots(F_{t_1}(\Z))\cdots ))$.

\begin{lem}\lemlabel{lower-domination}
   For any integer $r\ge 0$ and any $t_1,\ldots,t_r$,
   \[  F_{t_1,\ldots,t_r}(\Z) \subseteq F^{r}(\Z)   \]
\end{lem}

\begin{proof}
   Suppose that this is not the case, so that there is some
   minimum value of $r$ and $t_1,\ldots,t_r$ that provides a
   counterexample.  For $m\in\{1,\ldots,r\}$, let $A_m=F^{m}(\Z)$ and let
   $L_m=F_{t_1,\ldots,t_m}(\Z)$.  Since $r$ is minimum, $L_{r-1}\subseteq
   A_{r-1}$ and yet there is some $i\in L_r$ that such that $i\not\in
   A_r$.

   The fact that $i\not\in A_r$ implies that $X_i
   < \max\{X_{\cw_{A_{r-1}}(i)},X_{\ccw_{A_{r-1}}(i)}\}$.
   Suppose, without loss of generality, that $X_i < X_{\cw{A_{r-1}}(i)}$.
   Now, since $i\in L_r$, it must be that $X_i
   \ge X_{\cw{L_{r-1}}(i)}$.  But, since $L_{r-1}$ is
   $X$-safe, this contradicts \lemref{domination} and the assumption that
   $L_{r-1}\subseteq A_{r-1}$.
\end{proof}

\begin{lem}\lemlabel{lower-iid}
   Let $L$ be an $X$-safe set such that $\langle X_i:i\in L\rangle$ is iid.  Then, for any $t$, the sequence $\langle X_i: i\in F_t(L)\rangle$ is iid.
\end{lem}

\begin{proof}
   What to say here?
\end{proof}

To describe our sequence of subsets $(L_r: r\in\N)$, all that remains
is to describe the sequence of threshold values $(t_r:r\in\N)$ that is
used to obtain each set $L_r = F_{t_1,\ldots,t_r}(\Z)$.  Define the
set $L_0=\Z$.  By assumption, we know that $\langle X_i:i\in L_0\rangle$ is iud
over $[0,1]$.  For $r\ge 1$, \lemref{lower-iid} implies that $\langle X_i:i\in
L_{r-1}\rangle$ is iid over some distribution $\mathcal{D}_{r-1}$ and we take
$t_r$ to be the median of $\mathcal{D}_{r-1}$ so that, for $i\in L_{r-1}$,
$\Pr\{X_i > t_r\}=\Pr\{X_i < t_r\} = 1/2$.

\begin{lem}\lemlabel{lower-bound}
   For any $i\in\Z$ and any $r\in\N$, $\Pr\{i\in L_r\} = 1/4^r$.
\end{lem}

\begin{proof}
   It suffices to prove that $\Pr\{i\in L_r \mid i\in L_{r-1}\} = 1/4$
   since \[ \Pr\{i\in L_r\} = \prod_{k=1}^r \Pr\{i\in L_k\mid i\in L_{k-1}\} .\]

   Perform a relabelling of indices so that, for all $j\in\Z$,
   $Y_j=\cw^j_{L_{r-1}}(i)$. Consider the graph $G'$ described above and
   define the random variable $M_0$ as follows:  If no component of $G'$
   contains $i$ (because $X_i=Y_0<t_r$) then define $M_0=0$.  Otherwise,
   let $C_0$ be the component of $G'$ that contains $i$ and let $M_0$
   be the number of vertices in $C_0$.

   First note that $\Pr\{M_0=0\}=\Pr\{X_i<t_r\}=1/2$.
   For $m \in \N$, 
   \[
       \Pr\{M_0=m\} = m/2^{m+2}
   \]
   since $M_0=m$ precisely if there exists a $k\in\{0,\ldots,m-1\}$
   such that $Y_{-k-1}< t_r$, $Y_{-k+i}<t_r$ and $Y_j \ge t_r$ for all
   $j\in\{-k,\ldots,-k+m-1\}$.  Now we have,
   \begin{align*}
      \Pr\{i\in L_r\mid i\in L_{r-1}\} 
          & = \Pr\{\mbox{$M_0>0$ and $X_i=\max\{X_j:j\in C_0\}$}\} \\
          & = \sum_{m=1}^\infty \Pr\{X_i=\max\{X_j:j\in C_0\}\mid M_0=m\}\Pr\{M_0=m\} \\ 
          & = \sum_{m=1}^\infty (1/i)\Pr\{M_0=m\} \\
          & = \sum_{m=1}^\infty 1/2^{m+2} \\
          & = 1/4 \enspace . \qedhere
   \end{align*}
\end{proof}


Now, by \lemref{lower-domination} and \lemref{lower-bound}, $p_r=\Pr\{i\in
A_r\} \ge \Pr\{i\in L_r\}=1/4$ and we immediately obtain our lower bound:
\begin{cor}
   $p_r \ge 1/4^r$.
\end{cor}


\section{A Sharper Lower Bound}

Next we derive a sharper lower bound using refinements of ideas
from the previous section.

Let $L$ be an $X$-safe set such that $\langle X_i: i\in L\rangle$ is iid where
each element has distribution $\mathcal{D}$.  Again, let $G=(L,E)$ be the
graph with vertex set $L$ and edge set $E=\{(i,\cw_L(i)): i\in L\}$.
Fix a value $0<p<1$ and let $t=t(\mathcal{D},p)$ be chosen so that, for each $i\in L$,
$\Pr\{X_i\ge t\} = p$.  Consider the graph $G_{\ge t}$ induced the
elements in $L_{\ge t}$ and consider the graph $G_{<t}$ induced by the
elements in $L\setminus L_{\ge t}$.

Now, each component $C$ of $G_{<t}$ is a path $v_1,\ldots,v_k$.  For each
such component, we take the set 
\[ 
    \hat{C}_t=\{v_j : j\in\{2,\ldots,k-1\},\, X_{v_j}=\max\{X_{v_{j-1}},X_{v_j},X_{v_{j+1}}\}\} \enspace .
\]
That is, $\hat{C}$ are the indices of the internal local maxima in the
sequence $X_{v_1},\ldots,X_{v_k}$.  Let $\hat{F}_t(L)$ be the union of
$\hat{C}_t$ over all components $C$ in $G_{<t}$.

Let $Q_t(L)= F_t(L)\cup \hat{F}_t(L)$. It is straightforward to verify
that $Q_t(L)$ is $X$-safe.  However, the sequence $\langle X_i: i\in
Q_t(L)\rangle$ is certainly not iid.  

\begin{lem}\lemlabel{empty-lake}
  Select $t$ so that $\Pr\{X_i\ge t\}=p$.  Then,
  for any $i\in F_t(L)$, 
  \[
    \Pr\{\cw_{Q_t(L)}(i) = \cw_{F_t(L)}(i)\} =  
      1-\sum_{m=3}^\infty (1-p)^{m-1}p\left(1-\frac{2^{m-1}}{m!}\right)
  \]
\end{lem}

\begin{proof}
   The event we are studying occurs when $\hat{F}_t(L)$ contains
   no elements in the interval $[i,\cw_{F_t(L)}(i)]$.  Now, we know
   that there is at least one element $j$ in this interval with $X_j
   < t$.  Therefore $G_{<t}$ contains a component $C$ whose vertices
   $v_1,\ldots,v_m$ are in this interval. The component $C$ does not
   contribute any element to $\hat{F}_t(L)$ if and only if the sequence
   $X_{v_1},\ldots,X_{v_m}$ has no local maxima except possibly $X_{v_1}$
   and $X_{v_m}$. 

   Since $X_{v_1},\ldots,X_{v_m}$ is iid, the probability that it contains
   no local maxima is equal to the number of permutations with no local
   maxima divided by $m!$.  A straightforward counting argument using
   the binomial identity $\sum_{k=0}^{m-1}\binom{m-1}{k}=2^{m-1}$ shows
   that the number of permutations with no local maxima is $2^{m-1}$
   (see OEIS entry A059204 and references therein, for example \cite{X}).

   Now, the size of the component $C$ (i.e., the value of $m$)
   is also a random variable whose distribution is given by
   $\Pr\{|C|=m\}=(1-p)^{m-1}p$.  Letting $\mathcal{E}_C$ denote the event ``$C$  contributes at least one elements to $\hat{F}_t(L)$'', we have
   \[
       \Pr\{\mathcal{E}_C\} = \sum_{m=3}^\infty \Pr\{|C|=m\}\Pr\{\mathcal{E}_C\mid |C|=m\} = \sum_{m=3}^\infty (1-p)^{m-1}p\left(1-\frac{2^{m-1}}{m!}\right) \enspace . \qedhere
   \]
\end{proof}

Note that, using the identity $\sum_{k=0}^\infty a^k/(k!)=e^a-1$, one can derive a closed form for the probability in \lemref{empty-lake}:
\[
q = 
1-\frac{p e^{2(1-p)} + p - 2}{2 \, {\left(p - 1\right)}}
\]
(See the accompanying Sage workbook.)

\begin{lem}
  Let $t$ be chosen as in \lemref{empty-lake}. For any $i\in L$, $\Pr\{i\in F_t(Q_t(L))\}= (p-p^2)(1-q)$ and the sequence $\langle X_i:i\in F_t(Q_t(L))\rangle$ is iid.
\end{lem}


%\begin{lem}\lemlabel{magic}
%   Let $p=5/2-\sqrt{21/4}$, let $t^*$ be chosen so that $\Pr\{X\ge t^*\}=p$, 
%   and let
%   \[  c^*=4\sqrt{21}-18\approx0.330302779823 \enspace . \]
%   Then, 
%   \[  \Pr\{i\in Q_{t^*}(L)\} = c^* = 2\Pr\{i\in F_{t^*}(L)\}=2\Pr\{i\in \hat{F}_{t^*}(L)\} \enspace .
%   \]
%\end{lem}
%
%\begin{proof}
%  First note that $\Pr\{i\in Q_t(L)\}=\Pr\{i\in F_t(L)\} + \Pr\{i\in
%  \hat{F}_t(L)\}$, so it suffices to show that $\Pr\{i\in F_t(L)\} =
%  \Pr\{i\in \hat{F}_t(L)\}=c^*/2$.  Repeating the calculation in the
%  proof of \lemref{lower-bound}, we find that
%  \[
%      \Pr\{i\in F_{t^*}(L)\} = \sum_{m=1}^\infty p^m (1-p)^2 = p-p^2 = 2\sqrt{21}/9 \enspace .
%  \]
%  The calculation for $\hat{F}_{t*}$ is similar except that now, if $i$
%  appears in a component $C$ of $G_{<t^*}$ of size $m\ge 3$, it has
%  probability $(m-2)/3$ of taking part in $\hat{C}_{t^*}$.  This yields
%  \[
%     \Pr\{i\in \hat{F}_{t^*}(L)\} = \sum_{m=3}^\infty (1-p)^mp^2(m-2)/3
%     = -p^3/3 + p^2 - p + 1/3 = 2\sqrt{21}/9 \enspace . \qedhere
%  \]
%\end{proof}
%
%
%\begin{lem}
%   Let $p$, $t^*$, and $c^*$ be defined as in \lemref{magic}.  Then,
%   for any $i\in L$, $\Pr\{i\in F_{t^*}(G_{t^*}(L))\} = c^*/4$ and the
%   sequence $\langle X_i:i\in F_{t^*}(G_{t^*}(L))\rangle$ is iid.
%\end{lem}
%
%\begin{proof}
%   By \lemref{magic} we know that $\Pr\{i\in G_{t^*}(L)\}=c^*$. Let
%   $\mathcal{D}$ be the distribution of $X_i$ conditioned on $i$ being
%   in $G_{t^*}(L)$.  Again, by \lemref{magic} we see that $t^*$ is the
%   median of $\mathcal{D}$.  To be continued....
%\end{proof}
%
%
%\begin{thm}
%   $p_r\ge a^r$ for $a=\sqrt{\sqrt{21}-9/2}\approx 0.287359870121$
%\end{thm}
%



\section{The Upper Bound}

For $S\subseteq\Z$ a \emph{blocking} of $S$ is a partition $B$ of $S$
into sets of sizes 1 and 2 with the property that, if $\{i,j\}\in B$, then
$j\in\{\cw_S(i),\ccw_S(i)\}$.  We define $S[B] = \{\min P : P\in B\}$.  


\section{Remarks}

\bibliographystyle{plainurl}
\bibliography{anagram}

\end{document}


