\documentclass{patmorin}
\listfiles
\usepackage[utf8]{inputenc}
\usepackage{microtype}
\usepackage{amsthm,amsmath,amsopn,graphicx}
\usepackage{pat}
\usepackage[letterpaper]{hyperref}
\usepackage[table,dvipsnames]{xcolor}
\definecolor{linkblue}{named}{Blue}
\hypersetup{colorlinks=true, linkcolor=linkblue,  anchorcolor=linkblue,
citecolor=linkblue, filecolor=linkblue, menucolor=linkblue,
urlcolor=linkblue} 
\setlength{\parskip}{1ex}
\usepackage{wasysym}

\DeclareMathOperator*{\argmax}{arg\,max}
\DeclareMathOperator{\cw}{cw}
\DeclareMathOperator{\ccw}{ccw}

\title{\MakeUppercase{Notes and Questions on the Expected Performance 
   of the Franklin Leader Election Algorithm}%
   \thanks{This work was partly funded by NSERC and the Ontario Ministry of
    Research, Innovation and Science}}

\author{Probability and Combinatorics 2018}

%\usepackage[mathlines]{lineno}
%\linenumbers
%\setlength{\linenumbersep}{2.5cm}
%\rightlinenumbers
%\linenumbers
%\newcommand*\patchAmsMathEnvironmentForLineno[1]{%
%  \expandafter\let\csname old#1\expandafter\endcsname\csname #1\endcsname
%  \expandafter\let\csname oldend#1\expandafter\endcsname\csname end#1\endcsname
%  \renewenvironment{#1}%
%     {\linenomath\csname old#1\endcsname}%
%     {\csname oldend#1\endcsname\endlinenomath}}% 
%\newcommand*\patchBothAmsMathEnvironmentsForLineno[1]{%
%  \patchAmsMathEnvironmentForLineno{#1}%
%  \patchAmsMathEnvironmentForLineno{#1*}}%
%\AtBeginDocument{%
%\patchBothAmsMathEnvironmentsForLineno{equation}%
%\patchBothAmsMathEnvironmentsForLineno{align}%
%\patchBothAmsMathEnvironmentsForLineno{flalign}%
%\patchBothAmsMathEnvironmentsForLineno{alignat}%
%\patchBothAmsMathEnvironmentsForLineno{gather}%
%\patchBothAmsMathEnvironmentsForLineno{multline}%
%}


\newcommand{\question}[1]{\textbf{\color{red}Question:}~#1}

\DeclareMathOperator{\pw}{pw}

\newcommand{\eps}{\epsilon}



%\pagenumbering{roman}
\begin{document}
%\begin{titlepage}
\maketitle
%
\begin{abstract}
  We present upper and lower-bounds on the performance of an algorithm
  of Franklin for leader election on the ring when the node identifiers
  are independent uniform random variables in $[0,1]$.
\end{abstract}
%\end{titlepage}
%
%\tableofcontents
%
%\newpage


\section{Introduction}
\pagenumbering{arabic}

Franklin's leader election algorithm is a classic algorithm for finding
the maximum value on an $n$-node ring network (a cycle) whose nodes
are have distinct labels $x_0,\ldots,x_{n-1}$ in the order they appear
around the cycle.  The algorithm is initialized with all nodes in an
active state and proceeds in rounds.  During each round, each node
$i$ sends and its label $x_i$ to its nearest active neighbour $i'$ in
the clockwise direction and its nearest active neighbour $i''$ in the
counterclockwise direction.  At the same time node $i$ receives the values
of the labels $x_{i'}$ and $x_{i''}$.  Node $i$ then checks if $x_i <
\max\{x_{i'},x_{i''}\}$ and, if so, node $i$ becomes inactive and does
not partcipate in any subsequent rounds, except to forward messages on
behalf of active nodes.  The algorithm terminates when only one node
remains active.

This algorithm clearly terminates after at most $\log_2 n$ rounds because
at most one out of every two consecutive active nodes remains active after
each round.  If the labels $x_0,\ldots,x_{n-1}$ are independent uniformly
distributed random variables over the interval $[0,1]$, then it seems
plausible that the expected number of rounds in this case is $\log_3 n$.
This seems plausible because, during the first round, node $i$ remains
active if and only if $x_i=\max\{x_{i-1},x_i,x_{i-2}\}$ and this event
clearly has probability $1/3$. Thus, in this probabilistic setting,
the expected number of nodes that remain active after the first round
is $n/3$.

Unfortunately, this argument fails in subsequent rounds because,
after the first round, the labels of active nodes are no longer
independent. Furthermore, a careful calculation (see \secref{X}) shows
that the probability that a particular node $i$ remains active after 2
rounds approaches
\[
    p_2 = \frac{3e^4 - 48e^2 + 233}{384} \approx 0.1096868681 < 0.111\ldots = 1/9 \enspace .
\]
This is somewhat surprising and suggests that the Franklin leader election algorithm may perform better than expected when identifiers are random.  In particular, it is conceivable that the expected number of rounds in this setting could be $o(\log n)$.


\subsection{The Model and Results}

For a set $S\subseteq\Z$ and any $i\in\Z$, define $\ccw_S(i) =
\min\{j\in\N: i-j\in S\}$ and $\cw_S(i) = \min\{j\in\N: i+j\in S\}$.
Sometimes we will work with sets $S$ of integers modulo $n$, in which
case we use the same definitions using arithmetic modulo $n$.

To understand this process as $n\to\infty$, we consider the infinite
case in which the nodes consist of all integers $i\in \Z$ and the labels are
independent uniform random variables $X_i\in[0,1]$, for all $i\in\Z$.

The initial set $A_0$ of active nodes is defined to be $A_0=\Z$ and, at round $r\ge 1$ we determine the set $A_r$ of active nodes after round $r$ as 
\[
    A_r = \{i \in A_{r-1} : i > \max\{X_{\ccw_{A_{r-1}}(i)},
                                      X_{\cw_{A_{r-1}}(i)}\} \} 
\] 
We study the probability $p_r$ that $0\in A_r$.  Intuitively, one would
expect that $p_r=1/3^r$, but the we know that this is not even true for
$r=2$.  Relaxing this, we might hope to prove that $\lim_{r\to\infty}
p_r^{1/r} = c$ for some $c$ close to $1/3$.  However, \emph{apriori}, it is not obvious that this limit exists and, if it does, it may be $1/2$ or $0$.

Here we establish the following upper and lower bounds on $p_r$:

\begin{thm}
   $A^r \le p_r \le B^r$.
\end{thm}

We have not been able to establish the existence of the limit
$\lim_{r\to\infty} p_r^{1/r}$ but this result shows that, if
$\lim_{r\to\infty} p_r^{1/r}=c$, then $0 < A \le c\le  B < 1/2$.

In terms of the number of rounds required by Franklin's algorithm on
a finite ring, these results establish that the expected number of
rounds is at most $\log_{1/B} n + o(\log n)$ and at least $\log_{1/A}
n- o(\log n)$.  In particular, these results rule out the possibility
that the expected number of rounds is $o(\log n)$.

\section{Tools for Sequences}

The technical difficulty in studying Franklin's algorithm is that the
random variables $\{X_i:i\in A_r\}$ are not independent.  In order to
address this difficulty, we study different sets, $L_r\subseteq
A_r\subseteq U_r$ that have the property that the sets $\{X_i:i\in L_r\}$
and $\{X_i:i\in U_r\}$ are each independent sets of random variables, which
makes them easier to study. In this way we obtain upper and lower bounds because
\[
    \Pr\{i\in L_r\} \le \Pr\{i\in A_r\} \le \Pr\{i\in U_r\} \enspace .
\]
To make this work, we need some preliminary results on the structure of
the sets we are working with.

Let $X=\{X_i:i\in \Z\}$ be a set of distinct numbers indexed by integers.
We say that a subset $S\subseteq\Z$
is \emph{$X$-safe} if, for every $i\in S$ and $k= \cw_{S}(i)$:
\[
    \max\{X_i,X_k\} \ge \max\{X_m: m\in \{i,\ldots,k\}\}
\]

\begin{lem}\lemlabel{domination}
   Let $L\subseteq U\subseteq\Z$ be sets of integers where $L$
   is $X$-safe.  If $i\in L\cap U$, $j=\cw_{U}(i)$ with $X_j > X_i$,
   then for $k=\cw_{L}(i)$,
   \[
         X_k \ge X_j \enspace . 
   \]
   (And the same result holds if we replace all occurrences of $\cw$
   with $\ccw$.)
\end{lem}

\begin{proof}
   Since $L\subset U$, $j\in \{i,\ldots,k\}$.  Since $L$ is $X$-safe,
   $\max\{X_i,X_k\} \ge \max\{X_i,\ldots,X_k\} \ge X_j$.  Now we have
   that $\max\{X_i,X_k\} \ge X_j$, but $X_i < X_j$, so it must be that
   $X_k\ge X_j$.
\end{proof}

For any set $A\subset\Z$ and $X=\{X_i:i\in \Z\}$, define the \emph{Franklin function}
\[
   F_X(A) = \{ i\in A : X_{i} > \max\{X_{\cw_A(i)},X_{\ccw_A(i)}\} \}
\]
Notice that for any $r\ge 0$, $F^r_X(\Z)$ is $X$-safe.

\section{The Upper Bound}

For $S\subseteq\Z$ a \emph{blocking} of $S$ is a partition $B$ of $S$
into sets of sizes 1 and 2 with the property that, if $\{i,j\}\in B$, then
$j\in\{\cw_S(i),\ccw_S(i)\}$.  We define $S[B] = \{\min P : P\in B\}$.  

\section{The Lower Bound}

To obtain the lower bound $p_r \ge 1/4^r$, we describe a thresholded
version of the Franklin function that produces a subset of the set
produced by the Franklin function.  Let $L\subseteq\Z$ be an $X$-safe set.

Observe that, for any real value $t$, the \emph{threshold set}
$T_t(L)=\{i:X_i> t\}$ is also an $X$-safe set.  

Let $P$ be any partition of $L$ with the property that, for each $I\in
P$, if $i,j,k\in L$ with $i < k$ and $i,k\in I$, then $j\in I$. Then it
is straightforward to verify that the set
\[
     P(L) = \bigcup_{I\in P} \argmax_{i\in I}(X_i)
\]
is also an $X$-safe set.

Consider the graph $G=(L,E)$ where $E=\{(i,cw(i)): i\in L\}$. Remove
from $G$ all vertices $i\in L$ such that $X_i < t$ to obtain a graph
$G'$ that has a set $P$ of connected components.  Now consider the
\emph{$t$-thresholded Franklin function}
\[
    F_t(L) = \bigcup_{I\in P} \argmax_{i\in V(I)}(X_i)
\]
Note that, for any $t$, $F_t(L)$ is $X$-safe, since it can be written
as $F_t(L) = P(T_t(L))$.

\begin{lem}
   For any integer $r\ge 0$ and any $t_1,\ldots,t_k$,
   \[  F_{t_r}(F_{t_{r-1}}(\cdots(F_{t_1}(\Z))\cdots )) \subseteq F^{r}(\Z)   \]
\end{lem}

\begin{proof}
   Suppose that this is not the case, so that there is some
   minimum value of $r$ and $t_1,\ldots,t_r$ that provides a
   counterexample.  For $m\in\{k-1,k\}$, let $A_m=F^{m}(\Z)$ and let
   $L_m=F_{t_m}(\cdots(F_{t_1}(\Z)\cdots)$.  Since $r$ is minimum,
   $L_{r-1}\subseteq A_{r-1}$ and yet there is some $i\in L_r$ that such
   that $i\not\in A_r$.

   The fact that $i\not\in A_r$ implies that $X_i
   < \max\{X_{\cw_{A_{r-1}}(i)},X_{\ccw_{A_{r-1}}(i)}\}$.
   Suppose, without loss of generality that $X_i < X_j$ with $j =
   \cw_{A_{r-1}}(i)$.  Now, since $i\in L_r$, it must be that $X_i
   \ge X_k$, where $k=\cw_{L_{r-1}}(i)$.  But, since $L_{r-1}$ is
   $X$-safe, this contradicts \lemref{dominator} and the assumption that
   $L_{r-1}\subseteq A_{r-1}$.
\end{proof}

\begin{lem}\lemlabel{lower-iid}
   Let $L$ be an $X$-safe set such that the random variables $\{X_i:i\in
   L\}$ are iid.  Then, for any $t$, the random variables $\{X_i: i\in
   F_t(L)\}$ are iid.
\end{lem}

\begin{proof}
   What to say here?
\end{proof}

Now all that remains is to describe the sequence of threshold
values $t_1,\ldots,\ldots,t_r$ that is used to obtain the set $L_r =
F_{t_r}(F_{t_{r-1}}(\cdots(F_{t_1}(\Z))\cdots))$.  Define the set $L_0=\Z$.
By assumption, we know that $\{X_i:i\in L_0\}$ is iid uniformly over
$[0,1]$.  For $r\ge 1$, \lemref{lower-iid} implies that $\{X_i:i\in
L_{r-1}\}$ is iid over some distribution $\mathcal{D}_{r-1}$ and we take
$t_r$ to be the median of $\mathcal{D}_{r-1}$ so that, for $i\in L_{r-1}$,
$\Pr\{X_i > t_r\}=\Pr\{X_i < t_r\} = 1/2$.

\begin{lem}
   For any $i\in\Z$, $\Pr\{i\in L_r\} = 1/4^r$.
\end{lem}

\begin{proof}
   It suffices to prove that $\Pr\{i\in L_r \mid i\in L_{r-1}\} = 1/4$.  

   Perform a relabelling of indices so that, for all $j\in\Z$,
   $Y_j=\cw^j_{L_{r-1}}(i)$.  Consider the graph $G'$ described above,
   and consider the component, $C_0$, of this graph that contains
   $i$ (if any) and let $M_0$ be the number of vertices in $C_0$.
   With probability $1/2$, $X_i=Y_0 < t_r$, so no such component $C_0$
   exists, in which case we define $M_0=0$.

   For $m \in \N$, $\Pr\{M_0=m\} = i/2^{m+2}$
   since $M_0=m$ precisely if there exists a $k\in\{0,\ldots,m-1\}$
   such that $Y_{-k-1}< t_r$, $Y_{-k+i}<t_r$ and $Y_j > t_r$ for all
   $j\in\{-k,\ldots,-k+m-1\}$.  Now we have,
   \begin{align*}
      \Pr\{i\in L_r\mid i\in L_{r-1}\} 
          & = \Pr\{Y_0=\max\{Y_i:i\in C_0\}\} \\
          & = \sum_{m=1}^\infty \Pr\{Y_0=\max\{Y_i:i\in C_0\mid M_0=i\}\}\Pr\{M_0=m\} \\ 
          & = \sum_{m=1}^\infty (1/i)\Pr\{M_0=m\} \\
          & = \sum_{m=1}^\infty 1/2^{m+2} \\
          & = 1/4 \enspace . \qedhere
   \end{align*}
\end{proof}

Since $L_r\subseteq A_r$, we immediately obtain our lower bound:
\begin{cor}
   $p_r \ge 1/4^r$.
\end{cor}


\bibliographystyle{plainurl}
\bibliography{anagram}

\end{document}


